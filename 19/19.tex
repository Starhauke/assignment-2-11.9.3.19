% \iffalse
\let\negmedspace\undefined
\let\negthickspace\undefined
\documentclass[journal,12pt,twocolumn]{IEEEtran}
\usepackage{cite}
\usepackage{amsmath,amssymb,amsfonts,amsthm}
\usepackage{algorithmic}
\usepackage{graphicx}
\usepackage{textcomp}
\usepackage{xcolor}
\usepackage{txfonts}
\usepackage{listings}
\usepackage{enumitem}
\usepackage{mathtools}
\usepackage{gensymb}
\usepackage{comment}
\usepackage[breaklinks=true]{hyperref}
\usepackage{tkz-euclide} 
\usepackage{listings}
\usepackage{gvv}  
\usepackage{tikz}
\usepackage{circuitikz}
\usepackage{caption}

\def\inputGnumericTable{}                                
\usepackage[latin1]{inputenc}                 
\usepackage{color}                            
\usepackage{array}                            
\usepackage{longtable}                        
\usepackage{calc}                            
\usepackage{multirow}                      
\usepackage{hhline}                           
\usepackage{ifthen}                          
\usepackage{lscape}
\usepackage{amsmath}
\newtheorem{theorem}{Theorem}[section]
\newtheorem{problem}{Problem}
\newtheorem{proposition}{Proposition}[section]
\newtheorem{lemma}{Lemma}[section]
\newtheorem{corollary}[theorem]{Corollary}
\newtheorem{example}{Example}[section]
\newtheorem{definition}[problem]{Definition}
\newcommand{\BEQA}{\begin{eqnarray}}
\newcommand{\EEQA}{\end{eqnarray}}
\newcommand{\define}{\stackrel{\triangle}{=}}
\theoremstyle{remark}
\newtheorem{rem}{Remark}

\begin{document}
\title{}
\author{Sasa Mardi, EE23BTECH11222}
\date{}
\maketitle
\textbf{Question 11.9.3-19:} Find the sum of the products of the corresponding terms of the sequences $2, 4, 8, 16, 32$ and $128, 32, 8, 2, \frac{1}{2}$.\\
\textbf{Solution:}
Define the sequences as follows:\\
Sequence 1: \( x_1(n) = 2 \times 2^n \) with a common ratio of \( r_1 = 2 \).\\
Sequence 2: \( x_2(n) = 128 \times \left(\frac{1}{4}\right)^n \) with a common ratio of \( r_2 = \frac{1}{4}\).\\
Tables for both sequences:
\[
\begin{array}{|c|c|c|}
\hline
n & x_1(n) & x_2(n)\\
\hline
0 & 2 & 128 \\
1 & 4 & 32 \\
2 & 8 & 8 \\
3 & 16 & 2 \\
4 & 32 & \frac{1}{2} \\
\hline
\end{array}
\]
Table for the product of corresponding terms:
\[
\begin{array}{|c|c|c|c|}
\hline
n & x_1(n) & x_2(n) & x_1(n) \times x_2(n) \\
\hline
0 & 2 & 128 & 256 \\
1 & 4 & 32 & 128 \\
2 & 8 & 8 & 64 \\
3 & 16 & 2 & 32 \\
4 & 32 & \frac{1}{2} & 16 \\
\hline
\end{array}
\]
Sum of the products of corresponding terms:
\[
256 + 128 + 64 + 32 + 16 = 496
\]
So, the sum of the products of the corresponding terms of the sequences is \( 496 \).
\end{document}

